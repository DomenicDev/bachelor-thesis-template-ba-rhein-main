\section{Hauptteil}

\subsection{Analyse}

\blindtext

\subsection{Methodik}

\blindtext

\subsubsection{Methodik 1}

\blindtext[1]

\subsubsection{Methodik 2}

\blindtext[1]

\subsection{Umsetzung}

\blindtext

\subsection{Glossar}

The \Gls{latex} typesetting markup language is specially suitable 
for documents that include \gls{maths}.

\subsection{Zitate}

Das ist ein indirektes Zitat \cite{bieg2016finanzierung}. Dieser Satz wurde von einer bestimmten Seite zitiert \cite[S. 45-47]{bieg2016finanzierung}.

\subsection{Abbildungen}

\Cref{fig:logo} präsentiert das Commerzbank-Logo in seiner vollen Schönheit.

\begin{figure}[h]
    \centering
    \includegraphics[width=1.0\textwidth]{figures/official_logo_ba.jpg}
    \caption{Commerzbank-Logo}
    Eigene Darstellung
    \label{fig:logo}
\end{figure}

Und es geht weiter im Text.

\subsection{Abkürzungen}

Die \ac{HFU} ist eine ganz tolle Hochschule. Besser als die \ac{HFU} ist keine Hochschule in Baden-Württemberg!

\subsection{Fußnoten}

Es kann ja mal sein, dass eine Fußnote verwendet werden, so wie hier \footnote{Super! Diese Fußnote bezieht sich auf obigen Text.}. In der Regel werden diese aber nicht häufig gebraucht\footnote{vgl. \cite{bieg2016finanzierung}}.

\subsection{Tabellen}

Tabellen können entweder von Hand erstellt werden oder mit einem Tool. Beispielsweise bietet sich die Website \textit{https://www.tablesgenerator.com/} gut für solche Zwecke an. \Cref{tab:example_table} zeigt eine Beispieltabelle.

\begin{table}[htb]
    \centering
    \begin{tabular}{|c|c|}
        \hline
        Vorname & Chrisi  \\
        \hline
        Nachname & Cassisi \\
        \hline
    \end{tabular}
    \caption{Beispieltabelle}
    \label{tab:example_table}
\end{table}

\subsection{Zitate}

Folgend ein Zitat mit Einrückung:

\begin{quote}
\enquote{Das Leben kann nur in der Schau nach rückwärts verstanden, aber nur in der Schau nach vorwärts gelebt werden.} -- (Søren Kierkegaard, Philosoph)
\end{quote}

Und es geht weiter.

\subsection{Mathematik}

Mathe ist cool! Du kannst mathematische Ausdrücke in derselben Zeile verwenden, z.B. $y=x^2$. Für größere Ausdrücke kann auch ein ganzer Block verwendet werden:

\[ f(x) = \frac{\sqrt{2e^5}}{2} + 1\]